\section{Leitfaden für teilstrukturierte User Interviews}

\subsection{Einleitung}
Als Teil meiner Bachelorarbeit untersuche ich die Konzeption sowie Erstellung eines personalisierbaren Corona-Dashboards für Millennials. Dieses Interview hilft mir dabei drei relevante Fragestellungen in diesem Zusammenhang zu klären. Hierbei ist es wichtig zu erwähnen, dass Sie sich keine Sorgen um korrekte oder inkorrekte Aussagen machen müssen, die besten Antworten sind ihre persönlichen Meinung. Als Erstes sehen wir uns an, welche Informationen in Bezug auf Corona Sie für relevant halten und welche Datenvisualisierungen Sie bevorzugen. Zum Schluss halten wir noch die für Sie wichtigsten Personalisierungsmöglichkeiten fest. Bevor wir jedoch beginnen möchte ich Ihnen noch kurz den Fachbegriff Dashboard erläutern
\\

\textit{Fachbegriff Dashboard}
\\
Der Begriff Dashboard kommt aus der Automobilindustrie und bezeichnet die Frontkonsole (das Dashboard) Ihres Fahrzeugs. Auf dieser sind die wichtigsten Informationen wie Geschwindigkeit, Ölstand etc. auf einen Blick ersichtlich. Dashboards sind also eine Zusammenstellung von verschiedenen Visualisierungen, welche die für Sie relevanten Informationen auf einen Blick ersichtlich machen.\\


\textit{Administrativer Teil}
\\
Kommen wir nun noch kurz zum administrativen Teil. Wie Sie bereits im Vorfeld informiert worden sind, wird dieses Meeting aufgezeichnet. Dies beinhaltet dabei sowohl Ton als auch das Video auf Ihrem Computer. Diese Aufnahmen werden jedoch nicht veröffentlicht, sondern dienen lediglich als Grundlage für Auswertungszwecke im Rahmen dieser Arbeit.\\\\
\textbf{WICHTIG: AUFNAHME STARTEN}


\clearpage
\subsection{Hauptteil}

\textit{Hauptfragen}
\begin{itemize}
    \item Untergeordnete Forschungsfrage 1 und 2 wird via Online Formular beantwortet: \url{https://forms.gle/JdErDGAQ2YkwPZe57}
    \item Untergeordnete Forschungsfrage 3 wird mittels think-aloud Ansatz beantwortet, hierbei wird der Nutzer aufgefordert folgendes Dashboard zu besuchen \url{https://www.covid19.admin.ch/de/overview}
    \begin{itemize}
        \item Was hat Ihnen an diesem Dashboard gefallen?
        \item Was hat Ihnen an diesem Dashboard nicht gefallen?
        \item Was würden Sie sich von diesem Dashboard persönlich wünschen?
        \item Welche Visualisierungen haben Ihnen besonders gefallen und warum?
        \item Gibt es noch andere Visualisierungen welche Sie noch ergänzen möchten (Twitter Feed, Corona Richtlinien in der unmittelbahren Umgebung via GPS)?
        \item Würden Sie es bevorzugen, wenn Sie selbst ein eigenes Dashboard kreieren könnten, sprich eigene Visualisierungen hinzuzufügen?
        \item Ist es Ihnen wichtig, dass Design an Ihre eigenen Bedürfnisse anpassen zu können?
        \item (Nur wenn Anpassung Design wichtig) Welche Aspekte des Designs wollen Sie anpassen können (Farbgebung, Schrift, Titel der Visualisierungen)?
        \item Würden Sie ein leeres Dashboard bevorzugen, auf welchem Sie Ihre eigenen Visualisierungen platzieren können?
        \item Wünschen Sie sich Vorlagen, bzw. Vorschläge von bereits fertig gestellten Dashboards für bestimmte Zwecke (zum Beispiel Über den Schweregrad der Pandemie informieren)?
        \item Sind Ihnen Filterfunktionalitäten wichtig, wenn ja welche (Region, Zeit etc.)?
        \item Würden Sie gerne die Visualisierungen mit Freunden teilen können?
        \item Würden Sie eher eine Desktop Applikation, eine Website oder eine App bevorzugen und warum?
        \item Hätten Sie persönlich Interesste, an der Evaluation eines solchen Prototypen teilzunehmen?
    \end{itemize}
\end{itemize}


\subsection{Schlussteil}
\begin{itemize}
    \item Wünschen Sie noch etwas zu ergänzen oder etwas mitzuteilen?
\end{itemize}

\textbf{WICHTIG: EINWILLIGUNGSERKLÄRUNG PER E-MAIL SENDEN}