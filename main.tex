\documentclass[12pt, oneside]{article}
\usepackage{geometry}
\usepackage[utf8]{inputenc}
\usepackage{setspace}
\usepackage{times}
\usepackage{url}
\urlstyle{same}
\usepackage{multirow}
\usepackage[usegeometry]{typearea}
% Useful for checking layout
% \usepackage{showframe}
\usepackage{fancyhdr}
\usepackage{blindtext}
% Indenting the first sentence after section
\usepackage{indentfirst}
% Set language
\usepackage[german]{babel}
% Abbreviations
\usepackage{glossaries}
% For big tables which span multiple pages
\usepackage{longtable}
% Images
\usepackage{graphicx}
\graphicspath{ {./images/} }
% APA Citation Style
\usepackage[natbibapa]{apacite}
% Smaller font size for captions
\usepackage[font=small,labelfont=bf]{caption}
% Table related packages
\usepackage{booktabs}


\geometry{
 a4paper,
 left=30mm,
 top=25mm,
 right=25mm,
 bottom=20mm,
 footskip=15pt,
}

\setstretch{1.3} % Define Line Spacing
\renewcommand{\headrulewidth}{0pt} % Remove footer line

\pagestyle{fancy} % Allow for customizing header and footer
% Customize footer for page number location
\fancyhf{}
\fancyfoot{}
\fancyhead[R]{Yannick Hutter}
\fancyhead[L]{Bachelorthesis}
\fancyfoot[R]{\thepage}



 \makeglossaries

 \newglossaryentry{dsr}
 {
     name=DSR,
     description={Design science Research}
 }
 \newglossaryentry{foph}
 {
     name=FOPH,
     description={Federal Office of Public Health}
 }
 \newglossaryentry{fhgr}
 {
     name=FHGR,
     description={Fachhochschule Graubünden}
 }
 \newglossaryentry{svi}
 {
     name=SVI,
     description={Social Vulnerability Index}
 }
 \newglossaryentry{who}
 {
     name=WHO,
     description={World Health Organisation}
 }



\begin{document}
\pagenumbering{roman}
\begin{titlepage}
	\begin{center}
		\Huge
		\textbf{Bachelorthesis}
		
		\vspace{0.5cm}
		\LARGE
		Analyse und Implementierung eines personalisierbaren Corona dashboards für Millenials
		
		\vspace{1.5cm}
		\normalsize
		\textbf{Yannick Hutter}\\
		\textbf{Digital Business Management Klasse 18tz}\\
		\textbf{Talackerstrasse 8}\\
		\textbf{8887 Mels}\\
		\textbf{yannick.hutter@stud.fhgr.ch}\\

		
		\vfill
		Referrent: Daniel Klinkhammer\\
		Korefferent: Michael Burch\\
		
		\vspace{0.8cm}
		
		
		Digital Business Management\\
		Fachhochschule Graubünden\\
		Mels, April 2022
	\end{center}
\end{titlepage}

\clearpage
\section*{Abstract}
TODO


\clearpage
\tableofcontents
\listoffigures
\listoftables

\clearpage
\printglossaries



\clearpage
\pagenumbering{arabic}

\section{Einleitung}
TODO

\subsection{Stand der Forschung}
TODO

\subsection{Forschungsfrage}
TODO

\subsection{Methodische Vorgehensweise}
TODO

\clearpage
\section{Problemidentifizierung und Motivation}
TODO

\subsection{Dashboard – Ein Begriff mit Ursprung in der Automobilindustrie}
TODO

\subsection{Die wichtigsten Komponenten und Typen von Dashboards}
TODO

\subsection{Überblick über die bestehenden Corona Dashboards}
TODO

\subsection{Motivation – Erstellung eines personalisierbaren Corona Dashboards für Millennials}
TODO

\clearpage
\section{Identifikation der Ziele}
TODO

\subsection{Erstellung des Untersuchungsinstrumentes}
TODO

\subsection{Evaluation von Visualisierungstypen für Corona Dashboards}
TODO

\subsection{Identifikation von relevanten Personalisierungsmöglichkeiten in Bezug auf Dashboards}
TODO

\clearpage
\section{Design und Development}
TODO

\subsection{Design mittels Sketching}
TODO

\subsection{High-Fidelity Prototyp als Web Applikation}
TODO

\clearpage
\section{Auswertung}
TODO

\clearpage
\section{Fazit}
TODO

\clearpage
\section{Reflexion und Limitationen}
TODO


\clearpage
\bibliographystyle{apacite}
\urlstyle{rm}
\bibliography{main.bib}

\clearpage
\section*{Anhang}
\begin{table}[ht]
\begin{tabular}{@{}p{4cm}p{4cm}p{6.5cm}@{}}
\toprule
\textbf{Quelle}                                          & \textbf{Schlüsselwörter}        & \textbf{Artikel} \\ \midrule
\url{https:\\scholar.google.com}                         & covid dashboard                 & ~\citep{Dong.2020}         \\ \midrule
                                                         &                                 & ~\citep{Florez.2020}       \\ \midrule
                                                         &                                 & ~\citep{Berry.2020}        \\ \midrule
                                                         & user centered dashboards        & ~\citep{Francois.2021}     \\ \midrule
                                                         &                                 & ~\citep{Young.2020}        \\ \midrule
                                                         & customizable dasbhoards         & ~\citep{Roberts.2017}      \\ \midrule
\url{https:\\dl-acm.org}                                 & covid19 dashboard               & ~\citep{Vitale.}           \\ \midrule
                                                         & evaluating crisis dashboards    & ~\citep{Ivanov.2018}       \\ \midrule
                                                         & data dashboards                 & ~\citep{Maheshwari.}       \\ \midrule
                                                         &                                 & ~\citep{Beheshti.}         \\ \midrule
\url{https:\\google.com}                                 & covid dashboard evaluation      & ~\citep{Barbazza.}         \\ \midrule
                                                         & how user use covid19 dashboards & ~\citep{Ivankovic.2021}    \\ \bottomrule
\end{tabular}
\caption{\label{tab:research-protocol}Rechercheprotokoll (Eigene Darstellung)}
\end{table}
\clearpage


\subsection*{Zeitplan}

\begin{table}[ht]
\begin{tabular}{@{}p{13cm}p{2cm}@{}}
\toprule
\textbf{Tätigkeit}                                                                                & \textbf{Stichtag} \\ \midrule
Erstellung des Untersuchungsinstrumentes                                                          & 08.05.2022        \\ \midrule
Kapitel Einleitung fertig stellen                                                                 & 08.05.2022        \\ \midrule
Kapitel Problemidentifizierung und Motivation fertig stellen                                      & 15.05.2022        \\ \midrule
Durchführung der teilstrukturierten Interviews mit Hilfe des erstellten Untersuchungsinstrumentes & 29.05.2022        \\ \midrule
Abgabe Exposé                                                                                     & 22.05.2022        \\ \midrule
Kapitel Identifikation der Ziele fertig stellen                                                   & 05.06.2022        \\ \midrule
Implementierung Prototyp                                                                          & 26.06.2022        \\ \midrule
Kapitel Design und Development fertig stellen                                                     & 26.06.2022        \\ \midrule
Kapitel Fazit fertig stellen                                                                      & 03.07.2022        \\ \midrule
Kapitel Reflexion und Limitation fertig stellen                                                   & 10.07.2022        \\ \midrule
Korrekturlesung und Verbesserung                                                                  & 24.07.2022        \\ \midrule
Abgabe Thesis                                                                                     & 25.07.2022        \\ \bottomrule
\end{tabular}
\caption{\label{tab:time-table}Zeitplan (Eigene Darstellung)}
\end{table}


\clearpage
\section*{Eigenständigkeitserklärung}
Hiermit bestätigt der Verfasser, dass die vorliegende Arbeit selbstständig verfasst und keine anderen als die angegebenen Hilfsmittel benutzt wurden. Stellen der Arbeit, die dem Wortlaut oder dem Sinn nach anderen Werken entnommen sind, wurden unter Angaben der Quelle kenntlich gemacht.

\begin{figure}[ht]
	\includegraphics[width=6cm]{images/signature.png}
\end{figure}
Yannick Hutter, Mels am 01. Mai 2022

\end{document}