\section{Problemidentifizierung und Motivation von webbasierten Corona Dashboards}
Gemäss Peffers soll der Schritt \textit{Problem Identification and Motivation} dabei helfen, den Nutzen einer möglichen Lösung aufzuzeigen. Dies wiederum steigert zum einen die Motivation an der Arbeit selbst und sorgt zum anderen dafür, dass die Beweggründe für alle beteiligten Personen verständlich sind ~\citep[S. 52 + 55]{peffers}.


\subsection{Problemidentifikation}
Als Grundlage für die Problemidentifikation von webbasierten Corona Dashboards dient die Studie von Barbazza. Die Studie erhob die Erfahrungen und Eindrücke von 33 verschiedenen Teams im europäischen Raum, welche für die Konzeption und Umsetzung von Corona Dashboards verantwortlich waren. Im Rahmen der Studie wurden rund 80 Personen aus den 33 Teams befragt. Mit Hilfe von qualitativ durchgeführten Interviews wurden anschliessend verschiedene Kategoriesysteme gebildet. Im Kategoriesystem \textit{Nutzer} fallen folgende Problematiken auf ~\citep[S. 14 + 15]{barbazza}:
\begin{itemize}
    \item Keine Definierung einer dedizierten Zielgruppe für Dashboards, Dashboards wurden mehrheitlich für die breite Öffentlichkeit etc. konzipiert
    \item Limitierte Auffassung über den Informationsbedarf der Zielgruppe, kein systematischer Weg um an Nutzer Feedback zu gelangen
\end{itemize}

\subsection{Motivation – Erstellung eines personalisierbaren Corona Dashboards für Millennials}
Das Ziel dieser Arbeit ist die Konzeption und Erstellung eines Corona Dashboards für die Zielgruppe Millennials. Hierbei soll konkret ermittelt werden, welche Informationen sowie Visualisierungsarten für die Zielgruppe von Relevanz sind. Zudem soll ein besonderer Aspekt auf die Anpassbarkeit (Personalisierbarkeit) des Corona Dashboards gelegt werden. Die Lösung soll dazu dienen, den Erstellungsprozess von Dashboards in die Hände der Zielgruppe selbst zu legen. Nutzer können hierbei aus einem Katalog von Visualisierungen auswählen und ihr ganz persönliches Dashboard gestalten. Das erstellte Dashboard kann anschliessend an die relevanten Behörden (BAG) gesendet werden und soll so neue Ideen für Dashboard Designs antreiben sowie das Einholen von Feedback fördern.