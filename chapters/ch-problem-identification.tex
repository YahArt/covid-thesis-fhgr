\section{Problemidentifizierung und Motivation}
Gemäss Peffers soll der Schritt \textit{Problem Identification \& Motivation} dabei helfen, den Nutzen einer möglichen Lösung aufzuzeigen. Dies wiederum steigert zum einen die Motivation an der Arbeit selbst und sorgt zum anderen dafür, dass die Beweggründe für alle beteiligten Personen verständlich sind ~\citep[S. 52 + 55]{peffers}.


\subsection{Problemidentifikation}
Als Grundlage für die Problemidentifikation dient die Studie von Barbazza. Barbazza hat diverse Probleme in den Bereichen ``Users`` sowie ``Software`` identifiziert. Die wichtigsten Probleme waren hierbei die Entwicklung eines Dashboards für eine breite Zielgruppe, unzureichende Möglichkeiten um an Nutzerfeedback zu kommen sowie die Limitationen der eingesetzten Software in Bezug auf die Auswahl der Datenvisualisierungen und die Anpassungsmöglichkeiten ~\citep[S. 14-15]{barbazza}.

\subsection{Motivation – Erstellung eines personalisierbaren Corona Dashboards für Millennials}
Das Ziel dieser Arbeit ist die Konzeption und Erstellung eines Corona Dashboards für die Zielgruppe Millennials. Hierbei soll konkret ermittelt werden, welche Informationen sowie Visualisierungsarten für die Zielgruppe von Relevanz sind. Zudem soll ein besonderer Aspekt auf die Anpassbarkeit (Personalisierbarkeit) des Corona Dashboards gelegt werden. Die Lösung soll dazu dienen, den Erstellungsprozess von Dashboards in die Hände der Zielgruppe selbst zu legen. Nutzer können hierbei aus einem Katalog von Visualisierungen auswählen und ihr ganz persönliches Dashboard gestalten. Das erstellte Dashboard kann anschliessend an die relevanten Behörden (BAG) gesendet werden und soll so neue Ideen für Dashboard Designs antreiben sowie das Einholen von Feedback fördern.
