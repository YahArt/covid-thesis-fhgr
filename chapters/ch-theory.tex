\section{Dashboard - Alles auf einem Blick}
Das Ziel des vorliegenden Kapitel besteht darin einen Überblick über gängige Terminologien und Herangehensweisen in Bezug auf die Erstellung von Dashboards zu geben. 

\subsection{Begriffsdefinition Dashboard}
Unter dem Begriff Dashboard wird im Rahmen dieser Arbeit die Begriffsdefinition gemäss Stephen Few verwendet. Stephen beschreibt ein Dashboard als:
\begin{center}
\textbf{``Visual display of the most information needed to achieve one or more objectives which fits entirely on a single computer screen so it can be monitored at a glance.``} ~\citep[S. 26]{information_dashboard_design}  
\end{center}

Ein Dashboard erlaubt es somit der Zielgruppe alle relevanten Informationen auf einen Blick zu sehen und auf Basis dessen wichtige Entscheidungen zu treffen.

\subsection{Dashboard Design Prozess}
TODO


\subsubsection{Die 8 goledenen Regeln des Dashboard Designs}
TODO


\subsubsection{Visualization Pipeline}
TODO

\subsection{Web Technologien für die Gestaltung von Dashboards}
Gemäss Stephen Few gibt es verschiedene Arten Dashboards zu implementieren, die korrekte Lösung hängt hierbei stark vom Anwendungsfall ab. Im Rahmen der vorliegenden Arbeit hat sich der Author für die Umsetzung eines \textbf{webbasierten Dashboards} entschieden. Ein zentraler Vorteil von webbasierten Dashboards liegt darin, dass sehr einfach Visualisierungen von Echtzeitdaten möglich sind ~\citep[S. 27]{information_dashboard_design}. Ein weiterer ausschlaggebender Punkt ist die starke Verbreitung des Web Browsers, welcher auch auf jedem Smartphone verfügbar ist. Zudem können dank moderner Web Technologien anpassungsfähige Layouts implementiert werden.

\subsubsection{Angular}
TODO

\subsubsection{React}
TODO
